\documentclass{wx672article} % $HOME/texmf/tex/latex/wx672article.cls
\usepackage{wx672cjk}



% (c) 2002 Matthew Boedicker <mboedick@mboedick.org> (original author) http://mboedick.org
% (c) 2003-2007 David J. Grant <davidgrant-at-gmail.com> http://www.davidgrant.ca
% (c) 2008 Nathaniel Johnston <nathaniel@nathanieljohnston.com> http://www.nathanieljohnston.com
%
% (c) 2012 Scott Clark <sc932@cornell.edu> cam.cornell.edu/~sc932
%
%This work is licensed under the Creative Commons Attribution-Noncommercial-Share Alike 2.5 License. To view a copy of this license, visit http://creativecommons.org/licenses/by-nc-sa/2.5/ or send a letter to Creative Commons, 543 Howard Street, 5th Floor, San Francisco, California, 94105, USA.

%\documentclass[letterpaper,11pt]{article}
\newlength{\outerbordwidth}
\pagestyle{empty}
\raggedbottom
\raggedright
\usepackage[svgnames]{xcolor}
\usepackage{framed}
\usepackage{tocloft}


%-----------------------------------------------------------
%Edit these values as you see fit

\setlength{\outerbordwidth}{3pt}  % Width of border outside of title bars
\definecolor{shadecolor}{gray}{0.75}  % Outer background color of title bars (0 = black, 1 = white)
\definecolor{shadecolorB}{gray}{0.93}  % Inner background color of title bars


%-----------------------------------------------------------
%Margin setup

\setlength{\evensidemargin}{-0.25in}
\setlength{\headheight}{0in}
\setlength{\headsep}{0in}
\setlength{\oddsidemargin}{-0.25in}
\setlength{\paperheight}{11in}
\setlength{\paperwidth}{8.5in}
\setlength{\tabcolsep}{0in}
\setlength{\textheight}{9.5in}
\setlength{\textwidth}{7in}
\setlength{\topmargin}{-0.3in}
\setlength{\topskip}{0in}
\setlength{\voffset}{0.1in}


%-----------------------------------------------------------
%Custom commands
\newcommand{\resitem}[1]{\item #1 \vspace{-2pt}}
\newcommand{\resheading}[1]{\vspace{8pt}
  \parbox{\textwidth}{\setlength{\FrameSep}{\outerbordwidth}
    \begin{shaded}
\setlength{\fboxsep}{0pt}\framebox[\textwidth][l]{\setlength{\fboxsep}{4pt}\fcolorbox{shadecolorB}{shadecolorB}{\textbf{\sffamily{\mbox{~}\makebox[6.762in][l]{\large #1} \vphantom{p\^{E}}}}}}
    \end{shaded}
  }\vspace{-5pt}
}
\newcommand{\ressubheading}[4]{
\begin{tabular*}{6.5in}{l@{\cftdotfill{\cftsecdotsep}\extracolsep{\fill}}r}
		\textbf{#1} & #2 \\
		\textit{#3} & \textit{#4} \\
\end{tabular*}\vspace{-6pt}}
%-----------------------------------------------------------


\begin{document}

\begin{tabular*}{7in}{l@{\extracolsep{\fill}}r}
\textbf{\Large 蒲启元} & \textbf{\today} \\
  能自主独立的模仿着写一个简单的OS & pqy7172@gmail.com \\
  云南昆明 & 电话:18314555392 \\
  年龄:22 & 性别:男
\end{tabular*}
\\


%%%%%%%%%%%%%%%%%%%%%%%%%%%%%%
\resheading{教育背景}
%%%%%%%%%%%%%%%%%%%%%%%%%%%%%%
\vspace{2pt}
\begin{itemize}

\item 学校: 西南林业大学

  \begin{itemize}
  \item 学历:本科。
  \item 时间:2014-9 —— 2018-6。
  \item 专业:计算机科学与技术。
  \item 主修课程:数据结构,操作系统,组成原理等。
    
  \end{itemize}
  
\end{itemize}


%%%%%%%%%%%%%%%%%%%%%%%%%%%%%%
\resheading{项目介绍}
%%%%%%%%%%%%%%%%%%%%%%%%%%%%%%

\begin{enumerate}
\item RongOS操作系统开发。地址:https://github.com/Puqiyuan/RongOS。这个项目也是我的毕业设
  计,毕业论文地址https://github.com/Puqiyuan/RongOS/blob/master/doc/thesis/thesis.pdf。通
  过这个项目我对操作系统的各部分有了较为深刻的理解。

  值得一提的是论文及其中的插图都是我用LaTex排版的,有的图还非常复杂,比如进程管理的插图
  https://github.com/Puqiyuan/RongOS/blob/master/doc/thesis/figs/process-manage.pdf
  ,由此我的LaTeX排版能力应该也是不错的。
  
\item 清华清橙编程题练习。地址:https://github.com/Puqiyuan/Tsinsen\_ACM。这个项目是独立自
  主完成清橙编程题目。目前所完成的题目都是满分通过,其中某些题目的通过率较低,最低有16\%。
  所有题解代码最大的特色就是我只依赖了C标准库中少量的几个基本库函数比如printf,malloc等。
  当然实际生产产品时不提倡这样做,但在学习阶段我的观点还是较多的自己写代码,以训练自己的编码
  技巧。

  其它类似此的题解练习还有:https://github.com/Puqiyuan/URI\_ACM。这个开始的较近,所以暂时
  完成的题目不够多,难度也还不够大。不过会继续下去,编码是我的生活乐趣。

\item 其它几个值得一提的项目有https://github.com/Puqiyuan/CLRS\_Algorithms\_Implement,这个
  是实现算法导论里的算法,放在这主要证明我的确是热爱所学专业,热爱并愿意学习。学校方面并没有
  要求阅读英文原版的Introduction to Algorithm。

  接下来的两个小C程序再次用来证明我的编码能力。它们都是我完全独立写出来的。一个高精度(小
  数点前后各可达999位)浮点计算器:
  https://github.com/Puqiyuan/High\_Accuracy\_Float\_Calculator/blob/master/Calculator.c。
  
  操作系统中银行家算法的实现:
  https://github.com/Puqiyuan/OS\_Algorithms/blob/master/BankerAlgorithm/Programs/banker.c。
  
  
\end{enumerate}




%%%%%%%%%%%%%%%%%%%%%%%%%%%%%%
\resheading{自我评价}
%%%%%%%%%%%%%%%%%%%%%%%%%%%%%%
\vspace{2pt}


\hspace*{0.5cm} 具有较强的自学能力,大学期间培养了使用Google快速解决问题的能力,学习新东西较快,由于使用
Google英文,英文阅读沟通能力较强(CET6,考研英语一70分,满分100)。

\hspace*{0.5cm} 具有三年Debian Linux使用配置经验,熟悉Linux环境,有一定的Bash Shell编程经验。善于使用Makefile,
git,vim,emacs。编程,逻辑能力较优,如项目介绍中所体现。

\hspace*{0.5cm} 耐心毅力都不错,大学期间每周坚持十公里长跑。解决技术问题很多时候更靠耐心和
毅力。热爱学习,大学期间多次成绩排名前列。

\hspace*{0.5cm} 总结起来,大学期间,虽然没有多少实际商业项目经验,但是看的书较多,自己独立写的
代码不少,是同学眼中的学霸,最大的特点是爱钻研好学,在刚开始工作时,虽无多少经验,但拥有这
些基础能力,好的学习习惯,才有足够的后劲来持续学习发展。



%%%%%%%%%%%%%%%%%%%%%%%%%%%%%%
\resheading{个人荣誉}
%%%%%%%%%%%%%%%%%%%%%%%%%%%%%%
\vspace{2pt}
\begin{itemize}
\item 2014 — 2015年度校级三好学生。
\item 2015 — 2016年度省级三好学生。
\item 优秀毕业生。
\item 优秀毕业论文(设计)。
\end{itemize}



\end{document}








%%% Local Variables:
%%% mode: latex
%%% TeX-master: t
%%% End:

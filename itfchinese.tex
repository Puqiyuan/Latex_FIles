\documentclass[12pt,a4paper]{article}
\usepackage[top=3cm,bottom=3cm,left=3cm,right=3cm]{geometry}

\usepackage{graphicx}
\graphicspath{{./figs/}{../figs/}{./}{../}} %note that the trailing “/” is required

\usepackage{latexsym,pifont,units,amsmath,amsfonts,amssymb,marvosym}

\usepackage{indentfirst}
\usepackage[indentafter,pagestyles]{titlesec}

\usepackage{xcolor}

\usepackage{multicol,rotating,soul}
\setul{1.5pt}{.4pt}% 1.5pt below contents

\usepackage{xltxtra} %fontspec,xunicode are loaded here.
\defaultfontfeatures{Mapping=tex-text}
\setsansfont{DejaVu Sans}
\setmainfont{DejaVu Serif}

\usepackage{xeCJK}
\setCJKmainfont[BoldFont={WenQuanYi Zen Hei}, ItalicFont={WenQuanYi Zen Hei}]{SimSun}
\setCJKfamilyfont{hei}{WenQuanYi Zen Hei}
\setCJKfamilyfont{song}{SimSun}
\newcommand{\ziju}[1]{\renewcommand{\CJKglue}{\hskip #1}}

\usepackage{hyperref}

\usepackage{fancyhdr}
\pagestyle{fancy}

% use snippet 'im', 'minted' to add code blocks
\usepackage{minted}



\newcommand{\code}[1]{\texttt{\textcolor{violet}{#1}}}
\newcommand{\cfbox}[2]{%
  \colorlet{currentcolor}{.}%
  {\color{#1}\fbox{\color{currentcolor}#2}}%
}

\begin{document}
\title{ITF提议}
\author{蒲启元\\
  \emph{pqy7172@gmail.com}}
\maketitle{}
此文档是关于ITF的,那么什么是ITF?我会告诉你ITF是“信息技术是自由”的英文缩写(参见ITF提议英文
版)。或许你已经注意到ITF是一个无限递归式(同上)。那到底什么是ITF?

在我为你回答这个问题之前,我有几个小问题想问问你:
\begin{center}
  你执迷于编程吗?
  
  你爱探索内核?
  
  想成为一名黑客?
  
  要设计一个电路?
  
  被算法迷倒?
  
  像极客一样酷?
  
  想要实验室是你的家?
  
  准备一个IT竞赛?
  
  获得一些实际经验?
  需要些伙伴一起做个APP或者某个项目?
  
  ......
\end{center}

好的,如果你对于上面任何一个问题的回答是“yes”,那么ITF已经等你很久了,并且或许你也一直在寻
找她。但是现在,我们就要来场华丽的邂逅了。实际上,ITF是一个针对信息技术的沙龙。ITF的原则就
是自由与技术,就像她的名字所示——ITF。在这儿你可以:

\begin{center}
  遇见某个人他会给你一些灵感关于你百思不得其解的一些问题。
  
  自由的表达你的见解!其他ITFers都会倾听。
  对相同主题感兴趣的ITFers一起研究那些问题。
  
  你也会收获很多友谊。
  
  远不止这些......
  
  惊喜本来就是你该得的礼物,因为你加入了ITF。
  
\end{center}

下面是我的一些建议,所有建议都允许被讨论:

\begin{center}
  应该有一个固定的地点和时间来讨论。最好讨论问题的方式就是面对面。我推荐的地点是D225,D224
  或者是D227,时间定于每周日下午3点。
  
  不要称呼某人为学姐,学长之类的,名字就很完美。比如你可以叫我启元,如果像这样的方式你会不
  适应,请叫我蒲启元。因为ITF的氛围就是平等与自由,优先是绝对被禁止的。
  
  老师目前不被允许进入ITF,只为学生开放。
  
  出于对沙龙的质量考虑,进入ITF将会被严格控制,新ITFer只有得到旧ITFer的推荐才能进入ITF,当
  然你可以推荐你自己给任何一个旧ITFer,只要他们觉得你是专注于技术的。第一批ITFer的名字被列
  在了下面:
  
  杨洪竞
  
  赵仁贵
  
  蒋生飞
  
  杨建南
  
  蒲启元
  
  ITF欢迎任何旧ITFer推荐任何新加入者。但是旧ITFer需要给我发送一份新ITFer的基本信息,包括姓
  名,联系方式,专业和班级等,该信息只会用于登记和ITFers之间的交流,你可以选择任何你喜欢的
  方式给我基本信息,比如Email,Wechat,QQ。
  
  并且我要推荐一位新ITFer他的名字是:
  
  廖志明
  
\end{center}

你还在等什么呢?快来吧,ITF需要你!推荐你自己给旧ITFer,然后ITF将会给你想要的。

倡议人:第一批ITFers。

% \renewcommand{\abstractname}{Abstract}


\clearpage





\end{document}
% (setq-default TeX-master nil)


%%% Local Variables:
%%% mode: latex
%%% TeX-master: "contact"
%%% End:

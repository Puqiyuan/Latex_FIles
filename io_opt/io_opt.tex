\documentclass[scheme=plain]{ctexart}

\title{IO调度优化与Debug}
\author{Cauchy Pu}

\begin{document}

\maketitle

\begin{abstract}
本文介绍内核的IO调度,利用FIO工具来测试IO性能,并以此工具来复现IO相关的Bug并最终基于对内
核IO调度的分析来解决这个Bug,同时尝试优化申威内核的IO调度。
\end{abstract}

\clearpage
\pagenumbering{roman}
\tableofcontents
% \listoffigures{}
% \listoftables{}

\clearpage
\pagenumbering{arabic}
Obviously the amount of time depends on where the head was previously located and how
fortunate you are with the location of the sector on the platter: if it’s
directly under the head you do not need to wait, but if it just passed the head you have
to wait for a complete revolution. Even on the fastest 15k RPM disk that
takes 4 milliseconds (15,000 rotations per minute = 250 rotations per second, which means
one rotation is 1/250th of a second or 4ms). Admittedly that’s faster than
the sushi in my earlier analogy, but the chances are you will need to read or write a far
larger number of blocks than I can eat sushi dishes (and trust me, on a
good day I can pack a fair few away).

What about the next block? Well, if that next block is somewhere else on the disk, you
will need to incur the same penalties of seek time and rotational latency. We
call this type of operation a random I/O. But if the next block happened to be located
directly after the previous one on the same track, the disk head would
encounter it immediately afterwards, incurring no wait time (i.e. no latency). This, of
course, is a sequential I/O.

On the other hand, if you can perform the I/O sequentially you may be able to reduce the
IOPS requirement and increase the throughput, allowing the disk system to
deliver more data. I know of Oracle customers who spend large amounts of time and
resources carving up and re-ordering their data in order to allow queries to
perform sequential I/O. They figure that the penalty incurred from all of this preparation
is worth it in the long run, as subsequent queries perform better.
\end{document}

%%% Local Variables:
%%% mode: latex
%%% TeX-master: t
%%% End:

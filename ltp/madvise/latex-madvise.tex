\documentclass[scheme=plain]{ctexart}
\usepackage{minted}
\usepackage[a4paper, total={6in, 9in}]{geometry}
%\title{测试}
%\author{启元}

\begin{document}
%\maketitle
%\clearpage
%\pagenumbering{}
%\tableofcontents
% \listoffigures{}
% \listoftables{}

%\clearpage
\pagenumbering{arabic}
\begin{center}
  \begin{Huge}
    ltp问题分析之madvise
  \end{Huge}
\end{center}

\begin{center}
  \begin{huge}
    Cauchy Pu
  \end{huge}
\end{center}

\section{问题概述}
\label{sec:wtgs}
夹杂各种事情,终于ltp的madvise的分析报告出炉,自本次报告后我的工作重心可能就要转移了。

先看问题现象,amdvise共有几组测试例程,目前大部分都有问题。
\begin{minted}[frame=single]{bash}
sudo ./madvise01
CONF: MADV_HWPOISON is not supported
CONF: MADV_MERGEABLE is not supported
CONF: MADV_UNMERGEABLE is not supported
CONF: MADV_HUGEPAGE is not supported
CONF: MADV_NOHUGEPAGE is not supported
\end{minted}

\begin{minted}[frame=single]{bash}
sudo ./madvise02
CONF: MADV_WILLNEED is not supported
\end{minted}

\begin{minted}[frame=single]{bash}
sudo ./madvise06
CONF: memory cgroup needed
\end{minted}

\begin{minted}[frame=single, style=bw]{bash}
sudo ./madvise07
CONF: CONFIG_MEMORY_FAILURE probably not set in kconfig: EINVAL
\end{minted}

\begin{minted}[frame=single, style=bw]{bash}
sudo ./madvise09
CONF: '/sys/fs/cgroup/memory/' not present, CONFIG_MEMCG missing?
\end{minted}

\section{问题分析}
\label{sec:wtfx}
madvise是啥呢?先有个基础认识,摘抄一段man madvise:
\begin{quote}
  int madvise(void *addr, size\_t length, int advice):\\ \\
  The  madvise()  system call is used to give advice or directions to the kernel about the
  address range beginning at address addr and with size length bytes. In most cases, the
  goal of such advice is to improve system or application performance.
\end{quote}

看起来用途蛮大,用来给内核建议一段内存该怎么用,可以用来改善系统或应用程序的性能。但是申威
平台上报出了很多不支持的错误,根据前述经验,“不支持”可能不是真的“不支持”,还得仔细分析内核
配置、内核代码以及做实验验证每一种具体情况。

只有一个一个分析上述错误了。

\begin{enumerate}
\item madvise01
  
  这些MADV开头的其实都是advice的一种,下面来具体看看每种是什么意义。
  \begin{enumerate}
  \item MADV\_HWPOISON\\
    首先是man madvise的信息:
    \begin{quote}
      Poison the pages in the range specified by addr and length and handle subsequent
      references to those pages like a hardware memory corruption.  This operation  is
      available only for privileged (CAP\_SYS\_ADMIN) processes.  This operation may
      result in the calling process receiving a SIGBUS and the page being unmapped.\\
      
      This feature is intended for testing of memory error-handling code; it is available
      only if the kernel was configured with CONFIG\_MEMORY\_FAILURE.
    \end{quote}
    概要地说,这是一个feature,需要打开CONFIG\_MEMORY\_FAILURE配置。理论上,这个feature并
    没有太多的用处,只是把一个地址范围标记为poison,然后对这块地址的访问就像是在访问一块“内
    存出错”了的地址,这个feature可以用来测试“内存出错处理代码”。

    那么,什么叫“内存出错”,以及“内存出错处理代码”长啥样?暂且不表。

    现在尝试去打开MEMORY\_FAILURE,当然不是直接就打开这个内核选项的。帮助信息还是要看看:
    \begin{quote}
      Enables code to recover from some memory failures on systems with MCA recovery. This
      allows a system to continue running even when some of its memory has uncorrected
      errors. This requires special hardware support and typically ECC memory.
    \end{quote}

    有几个专业名词了,MCA recovery是啥?ECC memory是啥?稍微Google下,参考文献1、2给出了点
    有用信息。简单说就是,ECC memory是一种特别的内存,它可以在内存受到辐射等情况下时,出现
    了bit翻转也能纠错从而继续工作。为了配合ECC memory的这种特点,我们OS开发人员当然就有
    工作啦,比如现在ECC memory出错了,我们的内核是不是要提供个处理函数啊,这个就叫MCA
    recovery。\\
   
    那这跟MADV\_HWPOISON标志有什么关系嘛。\\
    
    嗯,其实就是,ECC memory出错(比如本来存的是比特0,后来翻转为1)概率其实是很小的,这主
    要在金融以及数据中心可能会考虑这种情况,或者是,你的计算机系统要运行在外太空,辐射较强,
    翻转机率提高。\\

    那我们写在内核里的“memory error-handling code”的代码就没法测了呀,咱也不能带着我们的
    Linux飞到外太空去不是,即使能这样做,在外太空翻转的概率也只是提高了,而不是一定。\\

    所以内核开发人员弄了个MEMORY\_FAILURE的feature,给某段内存标上这个标志后,访问这段内存
    就是“like a hardware memory corruption”,像是访问内存出错的地方。这样就可以触发我们的
    “memory error-handlingcode”了。\\

    说来说去,这个feature是为模拟测试硬件准备的,就是我们现在没有ECC memory这种“高级货”的
    时候也能写并测试“memory error-handling code”,暂时就先到这里吧。\\

    接下来先打开MEMORY\_FAILURE编译下。不出意外编译会有错,没错才是不正常。mm/madvise.c会
    报出有一些宏常数定义找不到,于是找到定义这些宏的文件,文件
    usr/in clude/asm-generic/mman-common.h即是。将mman-common.h拷贝一份到include/linux下,同时修
    改mm/madvise.c加上头文件\#include <linux/mman-common.h>,再编译,
    又会有另外的宏重复定义,在include/linux/mman-common.h中注释这些宏,于是新内核就可以编
    译出来测试了,发现本项测试通过。暂且到这儿吧,后面madvise07其实还有有问题,到那时再表。
    \textcolor{red}{最后要注意,这种修改其实是不“正规的”,不过为了往下测试验证也未尝不可,
      到用户真的有需要CONFIG\_MEMORY\_FAILURE,我们再和研究院沟通或他们做,或我们自己“正
      规”的来把这事顺手做了。}
  \item MADV\_MERGEABLE\\
    首先还是man madvise的信息了:
    \begin{quotation}
      Enable Kernel Samepage Merging (KSM) for the pages in the range specified by addr
      and length.  The kernel regularly scans those areas  of  user  memory that  have
      been marked as mergeable, looking for pages with identical content.  These are
      replaced by a single write-protected page (which is automatically copied if a
      process later wants to update the contentof the page).
            
      The MADV\_MERGEABLE and MADV\_UNMERGEABLE operations are available only if the kernel
      was configured with CONFIG\_KSM.
    \end{quotation}
    就是建议内核把相同的内容的page merge起来以节省内存。再结合参考信息3,此feature不涉及硬
    件信息,纯属打开内核配置的问题,经测试,打开CONFIG\_KSM配置后测试通过。
  \item MADV\_UNMERGEABLE\\
    与MADV\_MERGEABLE是配对的。
  \item MADV\_HUGEPAGE\\
    本项依旧是软件层面的事情,与硬件无关,内核社区自己已实现,打开内核配置
    CONFIG\_TRANSPARENT\_HUGEPAGE即可通过测试,与上述MADV\_MERGEABLE类似,不再赘述。
  \item MADV\_NOHUGEPAGE\\
    与MADV\_HUGEPAGE是配对的。
  \end{enumerate}
\item madvise02
  \begin{enumerate}
  \item MADV\_WILLNEED\\
    看看man信息:
    \begin{quote}
      Expect access in the near future.  (Hence, it might be a good idea to read some
      pages ahead.)
    \end{quote}

    把一个页面标记为MADV\_WILLNEED会建议内核提前把页面准备好,方便后面使用提高效率。这个特
    点没有要求任何内核特殊的配置。按道理不会打印is not supported啊。\\

    于是检查测试代码madvise02.c,这里打印了串:
\begin{minted}[frame=single, style=bw]{c}
if (tc->skip == 1) {
 tst_res(TCONF, "%s is not supported", tc->name);
 return;
}
\end{minted}
    而tc->skip在这里被改变成1:
\begin{minted}[frame=single, style=bw]{c}
if ((tst_kvercmp(3, 9, 0)) > 0 &&
				tc->exp_errno == EBADF)
 tc->skip = 1;
\end{minted}
    经查tst\_kvercmp会比较内核版本,我们的版本是4.19大于3.9,EBADF是说映射已经存在但不是一个
    文件。所有这一切均不是在确认madvise函数是否支持MADV\_WILLNEED,故认为这个not supported
    打印具有误导性,可以和社区沟通它究竟想表达什么。后面的实验验证会说明实际上
    MADV\_WILLNEED是支持的。
  \end{enumerate}
  
  另外值得一提的是,为了修复madvise01.c的MADV\_MERGEABLE is not supported问题,打开内核
  CONFIG\_KSM选项,发现madvise02出现了MADV\_MERGEABLE和MADV\_U-\\NMERGEABLE is not
  supported打印。\\

  实际上可以肯定,来自参考文献和内核代码实现的佐证,打开CONFIG\_KSM,madvise函数的
  MERGEABLE功能就会开起,这在前述madvise01分析中已经说明。\\

  那么为什么还会有这个打印呢?和MADV\_WILLNEED打印出来的流程类似,只不过这次是madvise02.c
  的下面代码修改了tc->skip:
\begin{minted}[frame=single, style=bw]{c}
if (access(KSM_SYS_DIR, F_OK) == 0)
 tc->skip = 1;
\end{minted}

  KSM\_SYS\_DIR被定义为/sys/kernel/mm/ksm,这在内核配置为CONFIG\_KSM时该文件自然存在。而
  F\_OK是测试文件的存在性,故条件满足。tc->skip被设置为1,进而not supported被打印,同样的
  情况,这个打印具有误导性,因为内核已经被配置为CONFIG\_KSM。
  
  
\item madvise06\\
  打印是“memory cgroup needed”,看样子是差什么东西,memory和cgroup,那差的这个是啥东西啊。\\
  
  看代码是这里打印出来的:
\begin{minted}[frame=single, style=bw]{c}
#define MNT_NAME "memory"
SAFE_MKDIR(MNT_NAME, 0700);
if (mount("memory", MNT_NAME, "cgroup", 0, "memory") == -1) {
 if (errno == ENODEV || errno == ENOENT)
  tst_brk(TCONF, "memory cgroup needed");
}
\end{minted}

  看样子是要看看mount了,老朋友man这么说:
  
  \begin{quote}
    int mount(const char *source, const char *target,
    const char *filesystemtype, unsigned long mountflags,
    const void *data):\\
    mount()  attaches the filesystem specified by source (which is often a pathname
    referring to a device, but can also be the pathname of a directory or file, or a dummy
    string) to the location (a directory or file) specified by the pathname in target.\\

    The  data  argument  is  interpreted  by  the  different filesystems.  Typically it is
    a string of comma-separated options understood by this filesystem.  See mount(8) for
    details of the options available foreach filesystem type.
  \end{quote}

  其中要注意的第二段,我已经故意摘出来了。我们要挂载的文件系统是cgroup,那么这个filesystem是
  怎么解释它的data的呢?\\

  简言之,/sys/fs/cgroup下有许多controller,而我们关心的是那个叫memory(就是data参数指明
  的)的controller,把它挂到由MNT\_NAME指明的路径下...等等,啥叫cgroup,controller又是啥啊?
  这个要看看参考文献4、5了。\\

  ls /sys/fs/cgroup/memory看下,嗯,确实没有这个目录,那它该怎么来呢?\\

  参考文献文献5讲了下,要打开CONFIG\_MEMCG配置,如此一试,果真如此。报错没有了,madvise06
  测试也过了。当然个中细节还要细看参考文献4、5啊。
\item madvise07
  
  madvise07要费点劲了。\\

  首先是没有前述的配置选项打开时,错误表现为前述问题概述节的madvise07贴图。打开
  CONFIG\_MEMORY\_FAILURE时,错误表现为:
\begin{minted}[frame=single, style=bw]{c}
FAIL: Did not receive SIGBUS on accessing poisoned page
\end{minted}

  怎么就收不到SIGBUS信号了呢?本情况下,多半是申威内核差点东西,因为前述提到
  MEMORY\_FAILURE需要特殊的内存类型,但是,这么点“东西”是啥呢?看来还需要细细探究一番。\\

  申威内核运行madvise07是上诉did not receive的表现,那Intel内核呢?于是在X86-64平台运行了
  下madvise07,结果正常,打印:
\begin{minted}[frame=single, style=bw]{c}
PASS: Received SIGBUS after accessing poisoned page
\end{minted}

  这说明X86-64上可以收到SIGBUS信号。那就看看madvise07.c了,关键的是这段(重组了下顺序):
\begin{minted}[frame=single, style=bw]{c}
//函数run_child
if (madvise(mem, msize, MADV_HWPOISON) == -1) {
 if (errno == EINVAL) {
  tst_res(TCONF | TERRNO,
  "CONFIG_MEMORY_FAILURE probably not set in kconfig");
   } else {
      tst_res(TFAIL | TERRNO, "Could not poison memory");
   }
  exit(0);
}

*((char *)mem) = 'd';
tst_res(TFAIL, "Did not receive SIGBUS on accessing poisoned page");

//另一个函数run
if (pid == 0) {
 run_child();
 exit(0);
}
SAFE_WAITPID(pid, &status, 0);
	if (WIFSIGNALED(status) && WTERMSIG(status) == SIGBUS) {
	 tst_res(TPASS, "Received SIGBUS after accessing poisoned page");
     return;
}
\end{minted}

  简单来说就是,父进程起了个子进程运行run\_child函数,而在run\_child函数里,调用了madvise
  把一段内存标记为MADV\_HWPOISON,随后语句*((char *)mem) = 'd'访问了这段内存。按道理,访问
  一段标记为MADV\_HWPOISON的内存应该被发送SIGBUS信号,从而子进程被杀死,父进程里打印出
  Received SIGBUS after accessing poisoned page。如果被杀死的子进程没有收到SIGBUS,也就是
  语句*((char *)mem) = 'd'访问并没有引起子进程异常退出,所以后面的一句:
\begin{minted}[frame=single, style=bw]{c}
tst_res(TFAIL, "Did not receive SIGBUS on accessing poisoned page");
\end{minted}
  会被正常执行,就是在申威平台看到的这句话了,它不会出现在X86-64平台。\\

  userland的原因如此,还是那个问题申威内核究竟差了什么,导致没能发送SIGBUS信号给用户进程?\\
  
  参考文献1说了句蛮有用的话:

  \begin{quote}
    The code consists of a the high level handler in mm/memory-failure.c, a new page
    poison bit and various checks in the VM to handle poisoned pages.
  \end{quote}

  那就稍微研究下mm/memory-failure.c咯。\\

  可以看到,函数memory\_failre正是内存出错时会被调用的处理函数。它向下会一步步调用到
  kill\_proc函数,而函数kill\_proc里会调用force\_sig\_mceerr或send\_sig\_mceerr函数给应用
  程序发送信号并杀死应用进程。那我们就来看看内核里这两个函数的实现,它们实现在signal.c里:
  
\begin{minted}[frame=single, style=bw, breaklines]{c}
int force_sig_mceerr(int code, void __user *addr, short lsb, struct task_struct *t)
{
	struct siginfo info;

	WARN_ON((code != BUS_MCEERR_AO) && (code != BUS_MCEERR_AR));
	clear_siginfo(&info);
	info.si_signo = SIGBUS;
	info.si_errno = 0;
	info.si_code = code;
	info.si_addr = addr;
	info.si_addr_lsb = lsb;
	return force_sig_info(info.si_signo, &info, t);
}

int send_sig_mceerr(int code, void __user *addr, short lsb, struct task_struct *t)
{
	struct siginfo info;

	WARN_ON((code != BUS_MCEERR_AO) && (code != BUS_MCEERR_AR));
	clear_siginfo(&info);
	info.si_signo = SIGBUS;
	info.si_errno = 0;
	info.si_code = code;
	info.si_addr = addr;
	info.si_addr_lsb = lsb;
	return send_sig_info(info.si_signo, &info, t);
}
\end{minted}

  终于看到心心念念的SIGBUS啊!就是这里确定给用户程序发送信号的。怎么验证这点呢?很简单啦,
  把它改成SIGSEGV这样的,再在X86-64平台运行madvise07,打印就变成了FAIL: Child killed by
  SIGSEGV,这正是因为我们精准改动要的效果哦。\\

  而在申威平台这些都有,但还是没有PASS打印,那就是看看调用了memory\_failure啦。这里就很关
  键了,是文件arch/x86/kernel/cpu/mce/core.c!这个文件arch/sw\_64/下是没有的。\\

  上述文件里的函数do\_machine\_check会调用memory\_failure。而这个do\_machine\_check函数实
  际是内存出错时的\textcolor{red}{异常}处理函数,异常产生时,会\textcolor{red}{自动触发}执
  行。至于异常产生怎么就自动执行函数了,说实话这对于内核或者现代IT基础来说很关键,不过在本
  文里并不关心,参考文献6给出了点信息,感兴趣的读者可以看看。\\

  总之,到这里真相大白了,申威内核缺少响应内存错误的句柄函数do\_machine\_check,如果用户有
  需要(幻想申威也能有这样的客户啊,搭载申威CPU飞向外太空?超级数据中心也用SW CPU,国际金融
  中心相中太湖旁的WIAT),\textcolor{red}{我们就可以随手做了,注意这并不依赖要ECC memory才能
    做},正如前文所料,这正是MEMORY\_FAILURE的存在意义。
    
\item madvise09

  根据提示首先是打开CONFIG\_MEMCG配置了,错误转变成了:
\begin{minted}[frame=single, style=bw]{c}
FAIL: Found corrupted page
\end{minted}

  被标记为MADV\_FREE的页,并不会立即释放,只有在遭遇内存压力时才会。而在标记为MADV\_FREE和
  真正释放时之间有页又再次被释放的话,这些被再次写的页不应该被释放,4.19.90的Intel内核遵循
  这个,但是申威4.19.90内核不遵循,这些被再次写过的页释放后填充0,与第二次写入的值不一样,
  所以认为是corrupted了。
  
\end{enumerate}

\section{实验验证}
\label{sec:syyz}
分析中已经指明了如何验证,包括打开相关配置,查找、修改内核代码来验证。

\section{解决方案}
\label{sec:jjfa}
\begin{enumerate}
\item MADV\_HWPOISON:配置MEMORY\_FAILURE打开,同时实现Intel下等价的异常函数
  do\_machine\_check。
\item MADV\_MERGEABLE:配置CONFIG\_KSM打开。
\item MADV\_HUGEPAGE:配置TRANSPARENT\_HUGEPAGE打开。
\item MADV\_WILLNEED:跳过测试,与社区沟通not supported打印误导。
\item 打开CONFIG\_MEMCG来支持cgroup的memory controller。
\item 以MADV\_FREE标志调用madvise时,注意再次被写的页也会被释放,终极解决方案需要修正
  cgroup的memory controller实现以使得符合标准。
\end{enumerate}

\section{参考文献}
\label{sec:ckwx}
\begin{enumerate}
\item https://www.kernel.org/doc/html/v4.18/vm/hwpoison.html
\item https://en.wikipedia.org/wiki/ECC\_memory
\item https://www.kernel.org/doc/html/v4.19/admin-guide/mm/ksm.html
\item https://man7.org/linux/man-pages/man7/cgroups.7.html
\item https://www.kernel.org/doc/Documentation/cgroup-v1/memory.txt
\item https://www.kernel.org/doc/html/latest/x86/exception-tables.html
\end{enumerate}

\end{document}

%%% Local Variables:
%%% mode: latex
%%% TeX-master: t
%%% End:
